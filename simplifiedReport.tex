\begin{verbatim}
cite<-function(key,bib){
mybib<-RefManageR::ReadBib("references.bib", check=FALSE)
bib<-mybib
paste0("[@",key,"]")
}
\end{verbatim}

\hypertarget{abstract}{%
\section{Abstract}\label{abstract}}

The satisfaction of the German course was the lowest as the gray line
shows the highest density (frequency of responses) of very dissatisfied
and dissatisfied ratings. The students in our fictitious data set were
most satisfied with the Chinese course as the blue line is the lowest
for very dissatisfied and ``dissatisfied'' ratings while the difference
between the courses shrinks for ``satisfied'' and very satisfied. The
Japanese language course is in-between the German and the Chinese
course.

\includegraphics[width=5.05263in,height=4.04211in]{media/image1.png}

\hypertarget{introduction}{%
\section{Introduction}\label{introduction}}

\hypertarget{nonspecific-filtering}{%
\subsection{Nonspecific filtering}\label{nonspecific-filtering}}

Let us load the dataset from the which we work on. In Chapter 1 you can
find a comprehensive description of the acute lymphoblastic leukemia
data that we use here.

\begin{verbatim}
data("ALL")
\end{verbatim}

First, we construct a list of samples from tumors of B-cells.

\begin{verbatim}
bcell = grep("^B", as.character(ALL$BT))
\end{verbatim}

The BCR/ABL translocation -- formally, t(9;22)(q34;q11), which is often
called the Philadelphia chromosome, producing a fusion gene consisting
of the BCR and the ABL1 genes -- is relatively prominent in acute
lymphocytic leukemias and of therapeutic relevance. Here, we focus on
the subset of ALL samples that harbor this translocation and contrast it
with the group of samples for which none of the common cytogenetic
aberrations (group NEG) was detected.

\begin{verbatim}
moltyp = which(as.character(ALL$mol.biol) %in% c("NEG", "BCR/ABL"))
\end{verbatim}

Let us now construct a new data object ALL\_bcrneg that contains only
those samples that fulfill these two conditions.

\begin{verbatim}
ALL_bcrneg = ALL[, intersect(bcell, moltyp)]
ALL_bcrneg$mol.biol = factor(ALL_bcrneg$mol.biol)
\end{verbatim}

The second line of the above code chunk cleans up the factor variable
ALL\_bcrneg\$mol.biol by removing the empty levels. Now, if we are going
to filter on the basis of variability, we might first want to make sure
that the variability is not dominated by its dependence on the mean
expression level. If it were, then selecting on the basis of variability
would be confounded with selection on the basis of absolute level. There
are good reasons, in essence due to the existence of probe-sequence
specific background and gain factor effects, not to use the absolute
level for gene selection. To check for an association we plot rowwise
means versus rowwise standard deviations and plot these together with a
smoothed estimate of their regression.

\begin{verbatim}
library(vsn)
meanSdPlot(ALL_bcrneg)
\end{verbatim}

\includegraphics[width=5.05263in,height=4.04211in]{media/image2.png}
From the plot, presuming that we decide that the relationship is not
very strong, we proceed. Our next step is to set aside those probe sets
with low variability.

In the code below, we set aside the 80\% lowest variability probe sets.
We choose such a high fraction because we want to limit the length of
the subsequent computations. The best choice for this fraction depends
on the array design and the biological samples, but in practice it will
usually be much lower.

\begin{verbatim}
sds = esApply(ALL, 1, sd)
sel = (sds > quantile(sds, 0.8))
ALLset1 = ALL_bcrneg[sel,]
\end{verbatim}

A potential drawback of this approach is found in situations where we
are interested in an experimental factor in which one group of samples
has few members. In this case, a gene which is differentially expressed
between that group and the other(s) may not have a large overall
standard deviation. How would you address this situation? At this point
you may want to try to look at some heatmaps of the data to see if there
are any obvious patterns. Consult the manual page of the function by
typing: ?heatmap.

\hypertarget{differential-expression}{%
\subsection{Differential expression}\label{differential-expression}}

In Bioconductor, the genefilter package allows you to easily select
genes using a variety of filters. Additionally, for some tests and
comparisons we have developed fast versions. These include rowttests,
which perform a t-test for every row in a gene expression matrix;
rowFtests, which does F-tests; and rowQ, which calculates a quantile for
each row. First, and perhaps easiest is to use a t-test (Dudoit et al.,
2002).

\begin{verbatim}
library(genefilter)
tt = rowttests(ALLset1, "mol.biol")
names(tt)
## [1] "statistic" "dm"        "p.value"
\end{verbatim}

Many practitioners have learned that small p-values do not always
correspond to genes for which there have been large changes. Let us look
at the so-called volcano plot.

\begin{verbatim}
plot(tt$dm, -log10(tt$p.value), pch=".",
     xlab = expression(mean~log[2]~fold~change),
     ylab = expression(-log[10](p)))
\end{verbatim}

\includegraphics[width=5.05263in,height=4.04211in]{media/image3.png}

\hypertarget{multiple-testing}{%
\subsection{Multiple testing}\label{multiple-testing}}

One of the subject areas that has received a great deal of attention is
that of multiple testing. We provide a brief introduction to the
functionality in the multtest package (Pollard et al., 2005). Many of
the algorithms in the multtest package depend on random permutations of
the samples. The number of permutations is controlled by the parameter
B. In the following, we call mt.maxT to perform a permutation test,
using the Welch statistic.

\begin{verbatim}
library("multtest")
cl = as.numeric(ALLset1$mol.biol=="BCR/ABL")
resT = mt.maxT(exprs(ALLset1), classlabel=cl, B=1000)
## b=10 b=20    b=30    b=40    b=50    b=60    b=70    b=80    b=90    b=100   
## b=110    b=120   b=130   b=140   b=150   b=160   b=170   b=180   b=190   b=200   
## b=210    b=220   b=230   b=240   b=250   b=260   b=270   b=280   b=290   b=300   
## b=310    b=320   b=330   b=340   b=350   b=360   b=370   b=380   b=390   b=400   
## b=410    b=420   b=430   b=440   b=450   b=460   b=470   b=480   b=490   b=500   
## b=510    b=520   b=530   b=540   b=550   b=560   b=570   b=580   b=590   b=600   
## b=610    b=620   b=630   b=640   b=650   b=660   b=670   b=680   b=690   b=700   
## b=710    b=720   b=730   b=740   b=750   b=760   b=770   b=780   b=790   b=800   
## b=810    b=820   b=830   b=840   b=850   b=860   b=870   b=880   b=890   b=900   
## b=910    b=920   b=930   b=940   b=950   b=960   b=970   b=980   b=990   b=1000  
ord = order(resT$index) ## the original gene order
rawp = resT$rawp[ord] ## permutation p-values
\end{verbatim}

The figure below shows the histogram of unadjusted permutation p-values
as given by the vector rawp. The high proportion of small p-values
suggests that indeed a substantial fraction of the genes is
differentially expressed between the two groups.

\begin{verbatim}
hist(rawp, breaks=50, col="#B2DF8A")
\end{verbatim}

\includegraphics[width=5.05263in,height=4.04211in]{media/image4.png}

In order to control the familywise error rate (FWER), that is, the
probability of at least one false positive in the set of significant
genes, mt.maxT used the permutation-based maxT procedure ofWestfall and
Young (1993). We obtain 34 genes with an adjusted p-value below 0.05:

\begin{verbatim}
sum(resT$adjp<0.05)
## [1] 34
\end{verbatim}

A comparison of this number to the height of the leftmost bar in the
histogram suggests that we are missing a large number of differentially
expressed genes. The FWER is a very stringent criterion, and in some
microarray studies, few or no genes may be significant in this sense,
even if many more are truly differentially expressed. A more sensitive
criterion is provided by the false discovery rate (FDR), that is, the
expected proportion of false positives among the genes that are called
significant.We can use the procedure of Benjamini and Hochberg (1995) as
implemented in multtest to control the FDR:

\begin{verbatim}
res = mt.rawp2adjp(rawp, proc = "BH")
sum(res$adjp[,"BH"]<0.05)
## [1] 209
\end{verbatim}

\hypertarget{moderated-test-statistics-and-the-limma-package}{%
\subsection{Moderated test statistics and the limma
package}\label{moderated-test-statistics-and-the-limma-package}}

We use the p-values for ranking genes, and do not advocate interpreting
them as true probabilities. Nevertheless, the results of a multiple
testing adjustment can be informative for choosing selection cut-offs.
Note that a more formal treatment would need to take into account the
multiple t-tests as well as the implicit testing of the nonspecific
filtering. A t-test analysis can also be conducted with functions of the
limma package (Smyth, 2004). First, we have to define the design matrix.
One possibility is to use an intercept term that represents the mean
log2 intensity of a gene across all samples (the first column in the
below matrix, consisting of 1s), and to encode the difference between
the two classes in the second column.

\begin{verbatim}
#library("limma")
design = cbind(mean = 1, diff = cl)
\end{verbatim}

Next a linear model is fitted for every gene by the function lmFit, and
an empirical Bayes moderation of the standard errors can be performed
with the function eBayes (Smyth, 2004). This employs information from
all genes to arrive at more stable estimates of each individual gene's
variance.

\begin{verbatim}
library(limma)
fit = lmFit(exprs(ALLset1), design)
fit = eBayes(fit)
\end{verbatim}

We can list the ten most differentially expressed genes using the
function topTable. The three probe sets with the lowest p-value all map
to the ABL1 gene which is part of the fusion gene product caused by the
t(9;22)(q34;q11) translocation and which is known to be over-expressed
and acting as a strong oncogene in acute lymphoblastic leukemia.

\begin{verbatim}
library("hgu95av2.db")
ALLset1Syms = unlist(mget(featureNames(ALLset1),env = hgu95av2SYMBOL))
topTable(fit, coef = "diff", adjust.method = "fdr",sort.by = "p", genelist = ALLset1Syms)
##                ID    logFC  AveExpr        t      P.Value    adj.P.Val         B
## 1636_g_at    ABL1 1.100012 9.196420 9.033955 4.879256e-14 1.232012e-10 21.293157
## 39730_at     ABL1 1.152527 9.000049 8.587774 3.877138e-13 4.894886e-10 19.341283
## 1635_at      ABL1 1.202675 7.897095 7.338622 1.228926e-10 1.034346e-07 13.905543
## 1674_at      YES1 1.427212 5.001771 7.050134 4.554138e-10 2.874800e-07 12.667763
## 40504_at     PON2 1.181029 4.244478 6.664739 2.571219e-09 1.298466e-06 11.032194
## 40202_at     KLF9 1.779378 8.621443 6.391779 8.622708e-09 3.628723e-06  9.889235
## 37015_at  ALDH1A1 1.032702 4.330511 6.242224 1.662442e-08 5.996667e-06  9.269402
## 32434_at   MARCKS 1.678550 4.466311 5.971866 5.375808e-08 1.696739e-05  8.161945
## 37027_at    AHNAK 1.348702 8.444161 5.805421 1.097066e-07 3.077880e-05  7.489382
## 37403_at    ANXA1 1.117721 5.086540 5.483354 4.265368e-07 1.077006e-04  6.210589
\end{verbatim}

When you compare the resulting p-value with those from the parametric
t-test, you will see that they are almost identical:

\begin{verbatim}
plot(-log10(tt$p.value), -log10(fit$p.value[, "diff"]),
     xlab = "-log10(p) from two-sample t-test",
     ylab = "-log10(p) from moderated t-test (limma)",
     pch=".")
abline(c(0, 1), col = "red")
\end{verbatim}

\includegraphics[width=5.05263in,height=4.04211in]{media/image5.png}

The result is shown in the above figure. Because of the large number of
samples, the empirical Bayes moderation is not so relevant here: in
these dataset the gene-specific variance can be well estimated from the
data of each gene.

\hypertarget{small-sample-sizes}{%
\subsection{Small sample sizes}\label{small-sample-sizes}}

However, the empirical Bayes moderation may be quite useful in cases
with fewer replicates. Let us draw a subsample with three arrays from
each group from our data:

\begin{verbatim}
subs = c(35, 65, 75, 1, 69, 71)
ALLset2 = ALL_bcrneg[, subs]
table(ALLset2$mol.biol)
## 
## BCR/ABL     NEG 
##       3       3
\end{verbatim}

We repeat the testing procedure in the same way as before.

\begin{verbatim}
tt2 = rowttests(ALLset2, "mol.biol")
fit2 = eBayes(lmFit(exprs(ALLset2), design=design[subs, ]))
\end{verbatim}

Below is a plot of the results above.

\begin{verbatim}
plot(-log10(tt2$p.value), -log10(fit2$p.value[, "diff"]),
     xlab = "-log10(p) from two-sample t-test",
     ylab = "-log10(p) from moderated t-test (limma)",
     pch=".")
abline(c(0, 1), col = "red")
\end{verbatim}

\includegraphics[width=5.05263in,height=4.04211in]{media/image6.png}

Let us have a look at a gene that has a small p-value in the normal
t-test but a large one in the moderated test.

\begin{verbatim}
g = which(tt2$p.value < 1e-4 &fit2$p.value[, "diff"] > 0.02)
\end{verbatim}

We plot its expression values and use different symbols and colors to
encode the sample classes.

\begin{verbatim}
sel = (ALLset2$mol.bio == "BCR/ABL")+1
col = c("black", "red")[sel]
pch = c(1,16)[sel]
plot(exprs(ALLset2)[g,], pch=pch, col=col,ylab="expression")
\end{verbatim}

\includegraphics[width=5.05263in,height=4.04211in]{media/image7.png}

\hypertarget{gene-annotation}{%
\subsection{Gene Annotation}\label{gene-annotation}}

After having obtained a list of reporters from a microarray experiment
and mapping them to their target genes, one will want to use the
annotation of the genes and gene products to better interpret the
experimental results. Often, it is beneficial to use gene annotation in
the course of the primary analysis, in order to narrow down the set of
data to be considered and ameliorate multiple testing problems, or in
order to explore specific biological hypotheses.

From the ALL dataset, we select those samples that were obtained from
tumors harboring either the BCR/ABL or the ALL1/AF4 translocation.

\begin{verbatim}
data(ALL)
types = c("ALL1/AF4", "BCR/ABL")
bcell = grep("^B", as.character(ALL$BT))
ALL_af4bcr = ALL[, intersect(bcell,which(ALL$mol.biol %in% types))]
ALL_af4bcr$mol.biol = factor(ALL_af4bcr$mol.biol)
\end{verbatim}

We want to apply a nonspecific filtering step in order to remove probe
sets that are likely to be noninformative.We use the function nsFilter
from the genefilter package for that purpose. The default measure used
by nsFilter for the variance filtering step is the IQR. This is a
reasonable choice as long as the sizes of the sample groups are
approximately similar. This is not the case for our BCR/ABL ALL1/AF4
subset, where the ALL1/AF4 positive group is much smaller:

\begin{verbatim}
table(ALL_af4bcr$mol.biol)
## 
## ALL1/AF4  BCR/ABL 
##       10       37
\end{verbatim}

For calculation of the IQR, 50\% of the most extreme values are
discarded as outliers, thus the measure of variance will be dominated
mainly by the much larger BCR/ABL positive group. We could address this
problem by using a non-robust measure of variance like the standard
deviation, however this would make the filtering more susceptible to
outliers. Instead, we will look at the range between more extreme
quantiles, here.

\begin{verbatim}
qrange <- function(x){
  diff(quantile(x, c(0.1, 0.9)))
}

library("genefilter")
filt_af4bcr = nsFilter(ALL_af4bcr, require.entrez=TRUE,require.GOBP=TRUE, var.func=qrange, var.cutoff=0.5)
ALLfilt_af4bcr = filt_af4bcr$eset
\end{verbatim}

Now, let us load the packages with the necessary tools and annotation
data

\begin{verbatim}
#library("Biobase")
#library("annotate")
#library("hgu95av2.db")
\end{verbatim}

Our first step is to use the function rowttests to carry out a two-group
comparison and to select the top 100 genes.

\begin{verbatim}
rt = rowttests(ALLfilt_af4bcr, "mol.biol")
names(rt)
## [1] "statistic" "dm"        "p.value"
\end{verbatim}

\hypertarget{working-with-go}{%
\subsection{Working with GO}\label{working-with-go}}

The Gene Ontology (GO) is a structured vocabulary of terms describing
gene products according to molecular function, biological process, and
cellular component (The Gene Ontology Consortium, 2000). The molecular
function of a gene product describes what it can do at the biochemical
level but without reference to where or when this activity might occur.
The biological process of a gene product describes a biological
objective to which the gene product contributes. The cellular component
ontology describes locations, at the levels of subcellular structures
and macromolecular complexes. Examples of cellular components include
nuclear inner membrane and the ubiquitin ligase complex.

\hypertarget{supervised-machine-learning}{%
\section{Supervised Machine
Learning}\label{supervised-machine-learning}}

Machine learning (ML) is typically divided into two separate areas,
supervised ML and unsupervised ML. The first of these is referred to as
classification in the statistics literature, and the second is referred
to as clustering. Both types of machine learning are concerned with the
analysis of datasets containing multivariate observations.

One issue that typically arises in ML applications to high-throughput
biological data is feature selection. For example, in the case of
microarray data one typically has tens of thousands of features that
were collected on all samples, but many will correspond to genes that
are not expressed. Other features will be important for predicting one
phenotype, but largely irrelevant for predicting other phenotypes. Thus,
feature selection is an important issue. Fundamental to the task of ML
is selecting a measure of similarity among (or distance between)
multivariate data points.

\hypertarget{supervised-machine-learning-check-list}{%
\section{Supervised machine learning check
list}\label{supervised-machine-learning-check-list}}

-Filter out features (genes) that show little variation across samples,
or that are known not to be of interest. If appropriate, transform the
data of each feature so that they are all on the same scale. -Select a
distance, or similarity, measure. What does it mean for two samples to
be close? Make sure that the selected distance embodies your notion of
similarity. -Feature selection: Select features to be used for ML. If
you are using cross-validation, be sure that feature selection according
to your criteria, which may be data-dependent, is performed at each
iteration. -Select the algorithm: Which of the many ML algorithms do you
want to use? -Assess the performance of your analysis. With supervised
ML, performance is often assessed using cross-validation, but this
itself can be performed in various ways.

The ALL dataset contains over 100 samples, for a variety of different
subtypes of the disease. In the code below we load the data, and then
subset to the particular phenotypes in which we are interested. The
specific information we need is to select those with B-cell ALL, and
then within that subset, those that are NEG and those that are labeled
as BCR/ABL. The last line in the code below is used to drop unused
levels of the factor encoding mol.biol.

\begin{verbatim}
data(ALL)
bcell = grep("^B", as.character(ALL$BT))
moltyp = which(as.character(ALL$mol.biol)%in% c("NEG", "BCR/ABL"))
ALL_bcrneg = ALL[, intersect(bcell, moltyp)]
ALL_bcrneg$mol.biol = factor(ALL_bcrneg$mol.biol)
\end{verbatim}

\hypertarget{nonspecific-filtering-of-features}{%
\subsection{Nonspecific filtering of
features}\label{nonspecific-filtering-of-features}}

Nonspecific filtering removes those genes that we believe are not
sufficiently informative for any phenotype, so that there is little
point in considering them further. For the purpose of this teaching
exercise, we used a very stringent filter so that the dataset is small
and the examples will run quickly; in practice you would probably use a
less stringent filter. We use the function nsFilter from the genefilter
package to filter for a number of different criteria. For instance, by
default it removes the control probes on Affymetrix arrays, which can be
identified by their AFFX prefix. We also exclude genes without Entrez
Gene identifiers, and as suggested above, we select the top 25\% of
genes on the basis of variability across samples.

\begin{verbatim}
ALLfilt_bcrneg = nsFilter(ALL_bcrneg, var.cutoff=0.75)$eset
head(exprs(ALLfilt_bcrneg))
##               01005    01010    03002    04007    04008    04010    04016    06002    08001
## 41654_at   8.828648 8.094511 8.168164 8.254469 7.361298 9.503099 8.603330 6.773515 9.295929
## 35430_at   5.855489 3.641874 5.017166 5.311914 4.778362 4.325899 6.393523 6.633259 5.880185
## 38924_s_at 8.424110 7.729307 8.644447 8.945631 8.216529 7.195125 9.337314 8.823942 8.881610
## 36023_at   7.405947 6.301844 7.136245 7.527066 6.148952 6.330935 7.471733 6.944525 7.638783
## 266_s_at   9.045327 7.578499 8.765199 8.972763 8.254930 9.395069 7.792077 8.246472 8.771152
## 37569_at   8.079626 7.767153 8.117210 8.220167 7.643456 8.014061 8.302689 7.488749 8.899068
##               08011    08012    08024    09008    09017    11005    12006    12007    12012
## 41654_at   8.386748 9.326383 7.927259 9.138667 8.006808 7.967930 7.727360 8.616821 8.074623
## 35430_at   6.044584 6.629995 5.842344 6.420538 5.536865 6.114543 6.342375 5.674307 5.259431
## 38924_s_at 9.097399 9.082794 8.353832 7.224571 7.755654 8.501485 8.247791 8.668081 8.232745
## 36023_at   7.111556 7.837012 7.490714 7.173974 7.250574 6.413746 7.416139 6.963664 7.347160
## 266_s_at   9.292926 8.701688 5.520708 9.219472 8.722878 9.231635 9.070559 8.473812 6.288550
## 37569_at   7.812161 8.042430 7.738814 8.707214 8.538109 8.298557 7.381348 7.528288 7.249217
##               12019    12026    14016    15001    15005    16009    20002    22009    22010
## 41654_at   9.302439 7.332120 7.914735 7.273544 7.620857 7.749546 8.507203 7.556782 7.777151
## 35430_at   4.868694 5.864238 4.924253 5.722033 6.677922 4.958387 4.724708 5.287356 6.180261
## 38924_s_at 8.753591 8.138303 7.697848 8.908445 7.957503 7.531842 8.598408 7.655558 7.476499
## 36023_at   7.168034 7.070591 7.124375 6.739134 7.105273 6.183825 6.208769 6.422143 7.142425
## 266_s_at   8.065411 7.279904 5.141673 8.315370 8.487607 6.135236 5.835822 4.775661 6.990659
## 37569_at   7.662015 6.878414 7.243849 7.483123 9.217584 6.376393 7.260814 7.311884 8.335432
##               22011    22013    24001    24008    24010    24011    24017    24018    24022
## 41654_at   8.777299 8.731828 8.479426 8.148210 7.932522 8.335359 9.517224 8.563869 8.336168
## 35430_at   7.081967 5.613332 6.475653 5.129776 6.026798 5.542433 5.189920 4.876395 6.070339
## 38924_s_at 9.053423 8.255573 8.862210 8.587378 8.680288 8.240342 8.063822 7.510568 7.590677
## 36023_at   7.775586 7.171030 8.193865 6.307482 7.185639 7.079452 6.497058 6.418974 7.335754
## 266_s_at   8.056750 6.605943 7.604823 7.926746 8.696195 8.364659 7.754378 6.930111 7.232982
## 37569_at   8.680554 7.155823 8.639928 7.404970 7.649956 7.107873 6.634875 7.901837 8.756573
##               25003    25006    26001    26003    27003    27004    28001    28005    28006
## 41654_at   8.234872 8.630322 8.541641 8.466201 7.782210 9.221615 7.841782 9.087692 8.373875
## 35430_at   5.917114 4.722375 6.520764 6.235439 4.330426 5.205263 4.657934 5.366176 5.028125
## 38924_s_at 8.038698 7.219289 8.829956 7.891159 7.446301 8.886631 7.458876 8.387229 8.253687
## 36023_at   6.873535 6.075753 6.668481 7.276040 7.127008 6.859286 6.405451 7.005145 6.919874
## 266_s_at   7.824575 5.372289 6.974280 8.974530 8.481343 9.476918 8.866697 9.198704 5.619280
## 37569_at   7.504818 7.256328 6.816550 8.721719 8.627308 9.023705 8.119095 9.175909 8.566743
##               28007    28019    28021    28023    28024    28031    28035    28036    28037
## 41654_at   9.467728 8.837256 8.997232 9.121741 9.871054 7.857103 9.218828 8.569648 9.106992
## 35430_at   4.393043 4.104417 4.957406 5.640488 3.372191 5.145946 3.784037 5.897394 4.516038
## 38924_s_at 8.027459 6.692071 7.709409 8.475243 8.983860 7.633971 7.620777 7.570270 6.994572
## 36023_at   7.367557 6.010747 6.650392 7.293290 7.076400 7.025873 6.456335 7.222923 5.987907
## 266_s_at   6.639194 8.815818 9.722175 9.209258 6.900522 8.762969 7.199150 8.512645 8.076551
## 37569_at   8.849892 7.703401 9.575725 8.776570 8.719523 8.212823 7.860119 8.890932 8.305911
##               28042    28043    28044    28047    30001    31011    33005    36002    37013
## 41654_at   8.665500 9.123309 8.988580 9.671188 8.990145 9.418143 9.464176 7.885616 8.413325
## 35430_at   5.779521 6.340348 5.736402 6.155322 4.573978 6.117581 6.308283 5.608931 5.855131
## 38924_s_at 7.930592 8.002395 7.391385 8.921583 7.596922 8.061912 7.726571 8.142278 8.173580
## 36023_at   6.973413 7.747994 7.353532 7.304685 6.560346 7.493523 6.492410 7.677477 7.696518
## 266_s_at   9.109580 9.249810 7.554509 9.582006 8.615381 9.240385 8.476273 9.543679 5.844497
## 37569_at   8.187730 8.869171 8.307514 8.877778 8.788335 9.358917 7.951865 8.071890 8.089461
##               43001    43004    43007    43012    48001    49006    57001    62001    62002
## 41654_at   7.732309 8.110521 8.992291 9.090736 8.280921 7.613821 7.351801 8.594811 9.070206
## 35430_at   4.829416 4.768723 7.044922 5.276924 4.877153 5.569176 6.557429 6.200918 6.070897
## 38924_s_at 8.236623 7.594109 8.702661 7.780634 7.793355 9.081649 8.197571 8.285420 7.499041
## 36023_at   7.572554 7.047133 6.682941 7.048038 7.665009 7.664315 7.531252 7.804634 7.976873
## 266_s_at   7.436804 8.814873 7.309128 9.408554 8.794465 9.024997 9.056086 9.424484 8.344995
## 37569_at   7.962245 8.150832 8.154709 8.604458 8.227035 7.376492 8.827597 8.797025 8.700340
##               62003    64001    64002    65005    68001    68003    84004
## 41654_at   8.457478 8.423355 9.109379 6.956484 8.731574 7.849030 8.960964
## 35430_at   5.928910 5.934606 6.563591 5.046739 3.900298 6.571250 6.372936
## 38924_s_at 8.015164 8.476753 8.160178 8.994477 7.414112 7.664263 8.940619
## 36023_at   6.951412 7.305877 6.367466 7.048391 6.584269 6.584391 8.762137
## 266_s_at   7.894858 7.475200 6.015826 6.737809 5.597120 8.954388 6.668957
## 37569_at   8.207598 7.669785 6.823858 8.069104 5.780545 8.220527 8.156519
\end{verbatim}

\hypertarget{feature-selection-and-standardization}{%
\subsection{Feature selection and
standardization}\label{feature-selection-and-standardization}}

Feature selection is an important component of machine learning.
Typically the identification and selection of features used for
supervised ML relies on knowledge of the system being studied, and on
univariate assessments of predictive capability. Among the more commonly
used methods are the selection of features that are predictive using
t-statistic and ROC curves (at least for two-sample problems). In order
to correctly assess error rates it is essential to accounted for the
effects of feature selection. If cross-validation is used then feature
selection must be incorporated within the cross-validation process and
not performed ahead of time using all of the data. A second important
aspect is standardization. For gene expression data the recorded
expression level is not directly interpretable, and so users must be
careful to ensure that the statistics used are comparable. This
standardization ensures that all genes have equal weighting in the ML
applications. In most cases this is most easily achieved by
standardizing the expression data, within genes, across samples. In some
cases (such as with a t-test) there is no real need to standardize
because the statistic itself is standardized. In the code segments below
we standardize all gene expression values. It is important that
nonspecific filtering has already been performed. We first write a
helper function to compute the rowwise IQRs for us.

\begin{verbatim}
rowIQRs = function(eSet) {
  numSamp = ncol(eSet)
  lowQ = rowQ(eSet, floor(0.25 * numSamp))
  upQ = rowQ(eSet, ceiling(0.75 * numSamp))
  upQ - lowQ
}
\end{verbatim}

Next we subtract the row medians and divide by the row IQRs. Again, we
write a helper function, standardize, that does most of the work.

\begin{verbatim}
standardize = function(x) (x - rowMedians(x)) / rowIQRs(x)
exprs(ALLfilt_bcrneg) = standardize(exprs(ALLfilt_bcrneg))
\end{verbatim}

\hypertarget{selecting-a-distance}{%
\section{Selecting a distance}\label{selecting-a-distance}}

The dist function computes the distance between rows of an input matrix.
We want the distances between samples, thus we transpose the matrix
using the function t. The return value is an instance of the dist class.
Because this class is not supported by some R functions that we want to
use, we also convert it to a matrix.

\begin{verbatim}
eucD = dist(t(exprs(ALLfilt_bcrneg)))
eucM = as.matrix(eucD)
\end{verbatim}

We next visualize the distances using a heatmap. In the code below we
generate a range of colors to use in the heatmap. The RColorBrewer
package provides a number of different palettes to use and we have
selected one that uses red and blue. Because we want red to correspond
to high values, and blue to low, we must reverse the palette.

\begin{verbatim}
library("RColorBrewer")
hmcol = colorRampPalette(brewer.pal(10, "RdBu"))(256)
hmcol = rev(hmcol)
heatmap(eucM, sym=TRUE, col=hmcol, distfun=as.dist)
\end{verbatim}

\includegraphics[width=5.05263in,height=4.04211in]{media/image8.png}

\hypertarget{actual-machine-learning}{%
\subsection{Actual Machine Learning}\label{actual-machine-learning}}

We start by looking at the k nearest neighbors (KNN) and diagonal linear
discriminant analysis (DLDA) methods, because they are conceptually
simple and serve well to demonstrate most of the general principles. It
is easiest to understand most supervised ML methods in the setting where
one has both a training set on which to build the model, and a test set
on which to test the model. We begin by artificially dividing our data
into a test and training set. Such a dichotomy is not actually that
useful and in practice one tends to rely on cross-validation, or other
similar schemes.

\begin{verbatim}
library(MLInterfaces)
Negs = which(ALLfilt_bcrneg$mol.biol == "NEG")
Bcr = which(ALLfilt_bcrneg$mol.biol == "BCR/ABL")
S1 = sample(Negs, 20, replace=FALSE)
S2 = sample(Bcr, 20, replace = FALSE)
TrainInd = c(S1, S2)
TestInd = setdiff(1:79, TrainInd)
\end{verbatim}

In the ALL dataset, some features are not likely to be predictive of the
phenotypes of interest, and so we now want to explore what happens if we
instead select genes that are able to discriminate between those with
BCR/ABL and those samples labeled NEG. We use the t-test to select
genes; those with small p-values for comparing BCR/ABL to NEG are used.
Although it is tempting to use all the data to do this selection, that
is not really a good idea as it tends to give misleadingly low values
for the error rates. You can, and probably should, use attenuated
t-tests, and you can select the ones to use by the observed p-value.
But, these approaches would complicate the exposition further, so we
simply select those 50 genes with the most extreme t-statistics. In the
code below, we compute the t-tests on the training set, then sort them
from largest to smallest, and then obtain the names of the 50 that have
the largest observed test statistics.

\begin{verbatim}
Traintt = rowttests(ALLfilt_bcrneg[, TrainInd], "mol.biol")
ordTT = order(abs(Traintt$statistic), decreasing=TRUE)
fNtt = featureNames(ALLfilt_bcrneg)[ordTT[1:50]]
\end{verbatim}

Repeat this exercise on the whole dataset. How many of the genes
selected on the training set were also selected when you used the whole
dataset? Now we can see how well the different machine learning
algorithms work when the features have been selected to help
discriminate between the two groups.

\hypertarget{knn}{%
\subsection{KNN}\label{knn}}

\begin{verbatim}
BNf = ALLfilt_bcrneg[fNtt,]
knnf = MLearn( mol.biol ~ ., data=BNf, knnI(k=1,l=0),TrainInd)
## [1] "mol.biol"
c_mat <- confuMat(knnf)
c_mat 
##          predicted
## given     BCR/ABL NEG
##   BCR/ABL      14   3
##   NEG           1  21
\end{verbatim}

\hypertarget{random-forest}{%
\subsection{Random Forest}\label{random-forest}}

\begin{verbatim}
library("randomForest")
set.seed(123)
rf1 = MLearn( mol.biol~., data=ALLfilt_bcrneg,randomForestI, TrainInd, ntree=1000, mtry=55,importance=TRUE)
## [1] "mol.biol"
trainY = ALLfilt_bcrneg$mol.biol[TrainInd]
confuMat(rf1, "train")
##          predicted
## given     BCR/ABL NEG
##   BCR/ABL      20   0
##   NEG           0  20
confuMat(rf1, "test")
##          predicted
## given     BCR/ABL NEG
##   BCR/ABL      12   5
##   NEG           1  21
\end{verbatim}

Next we use a much smaller value of mtry so that we can compare the
results.

\begin{verbatim}
rf2 = MLearn( mol.biol~., data=ALLfilt_bcrneg,randomForestI, TrainInd, ntree=1000, mtry=10,importance=TRUE)
## [1] "mol.biol"
\end{verbatim}

We can use the prediction function to assess the ability of these two
forests to predict the class for the test set. For each model we show
the confusion matrix for both the training and test sets. Naturally the
error rates are much smaller (zero in both cases) for the training set.

\hypertarget{feature-selection}{%
\subsection{Feature selection}\label{feature-selection}}

One of the nice things about the randomforest technology is that it
provides an indication of which variables were most important in the
classification process. The specific definitions of these measures are
provided in the manual page for the importance function, which can be
used to extract the measures from an instance of the randomForest class.
These features can be compared to those selected by t-test or selected
by some other means. In the next two code chunks we plot the variable
importance statistics for the two random forests. The output is shown in
Figure.

\begin{verbatim}
opar = par(no.readonly=TRUE, mar=c(7,5,4,2))
par(las=2)
impV1 = getVarImp(rf1)
plot(impV1, n=15, plat="hgu95av2", toktype="SYMBOL")
\end{verbatim}

\includegraphics[width=5.05263in,height=4.04211in]{media/image9.png}

\begin{verbatim}
par(opar)
#library(caret)
#imp<-varImp(rf1)
#ggplot(imp)
\end{verbatim}

\hypertarget{confusion-table}{%
\section{Confusion Table}\label{confusion-table}}

\begin{verbatim}
##   TClass PClass    Y
## 1      1      1  235
## 2      1      0   34
## 3      0      1  248
## 4      0      0 2618
\end{verbatim}

The following is the confusion table:

\# Confusion Matrix

\includegraphics[width=5.05263in,height=4.04211in]{media/image10.png}

Confusion Matrix Plot

If a diagnosis/test turns out positive ,the likelihood that the tested
patient has disease X is PPV Whereas If a diagnosis/test turns out
negative ,the likelihood that the tested patient don't have disease X is
NPV with, Risk ratio of risk\_ratio 95\% CI of between lci\_risk\_ratio
and uci\_risk\_ratio.

\hypertarget{survey-table}{%
\section{Survey Table}\label{survey-table}}

The ldat data set has only two columns: a column labeled \emph{Course}
which has three levels (\emph{German}, \emph{Japanese}, and
\emph{Chinese}) and a column labeled \emph{Satisfaction} which contains
values from 1 to 5 which represent values ranging from \emph{very
dissatisfied} to \emph{very satisfied}.

A plot of cummulative satisfaction for all groups in the survey data is
shown below.

\includegraphics[width=5.05263in,height=4.04211in]{media/image11.png}

The satisfaction of the German course was the lowest as the gray line
shows the highest density (frequency of responses) of \emph{very
dissatisfied} and \emph{dissatisfied} ratings. The students in our
fictitious data set were most satisfied with the Chinese course as the
blue line is the lowest for \emph{very dissatisfied} and
``dissatisfied'' ratings while the difference between the courses
shrinks for ``satisfied'' and \emph{very satisfied}. The Japanese
language course is in-between the German and the Chinese course.

\hypertarget{pie-charts}{%
\subsection{Pie charts}\label{pie-charts}}

Most commonly, the data for visualization comes from tables of absolute
frequencies associated with a categorical or nominal variable. The
default way to visualize such frequency tables are pie charts and bar
plots. In a first step, we modify the data to get counts and
percentages.

A pie chart of percentage distribution by groups.

\includegraphics[width=5.05263in,height=4.04211in]{media/image12.png}

If the slices of the pie chart are not labelled, it is difficult to see
which slices are smaller or bigger compared to other slices. This
problem can easily be avoided when using a bar plot instead. This issue
can be avoided by adding labels to pie charts. The labeling of pie
charts is, however, somewhat tedious as the positioning is tricky. Below
is an example for adding labels without specification.

\hypertarget{pie-plots-separated-into-groups}{%
\subsection{Pie plots Separated into
Groups}\label{pie-plots-separated-into-groups}}

Now that we have created the data, we can plot separate pie charts for
each course.

\includegraphics[width=5.05263in,height=4.04211in]{media/image13.png}

\hypertarget{bar-plots}{%
\subsection{Bar plots}\label{bar-plots}}

Like pie charts, bar plot display frequency information across
categorical variable levels.

\includegraphics[width=5.05263in,height=4.04211in]{media/image14.png}

Compared with the pie chart, it is much easier to grasp the relative
size and order of the percentage values which shows that pie charts are
unfit to show relationships between elements in a graph and, as a
general rule of thumb, should be avoided.

\hypertarget{grouped-bar-plots}{%
\subsection{Grouped bar plots}\label{grouped-bar-plots}}

Bar plots can be grouped which adds another layer of information that is
particularly useful when dealing with frequency counts across multiple
categorical variables. But before we can create grouped bar plots, we
need to create an appropriate data set.

\includegraphics[width=5.05263in,height=4.04211in]{media/image15.png}

Bar plots are particularly useful when visualizing data obtained through
Likert items. As this is a very common issue that empirical researchers
face. There are two basic ways to display Likert items using bar plots:
grouped bar plots and more elaborate scaled bar plots.

Although we have seen above how to create grouped bar plots, we will
repeat it here with the language course example used above when we used
cumulative density line graphs to visualize how to display Likert data.

In a first step, we recreate the data set which we have used above. The
data set consists of a Likert-scaled variable (Satisfaction) which
represents rating of students from three courses about how satisfied
they were with their language-learning course. The response to the
Likert item is numeric so that ``strongly disagree/very dissatisfied''
would get the lowest and ``strongly agree/very satisfied'' the highest
numeric value.

Again, we can also plot separate bar graphs for each class by specifying
``facets''.

\includegraphics[width=5.05263in,height=4.04211in]{media/image16.png}

\includegraphics[width=5.05263in,height=4.04211in]{media/image2.png}

\includegraphics[width=5.05263in,height=4.04211in]{media/image17.png}

\includegraphics[width=6.07011in,height=4.25741in]{media/image18.png}

\hypertarget{references}{%
\section{References}\label{references}}